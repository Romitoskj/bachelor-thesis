\begin{abstract}
Questa tesi descrive lo sviluppo e l'integrazione di un sensore Bluetooth nell'applicazione GeneroCity, un'applicazione di smart parking progettato dal Gamification Lab della Sapienza Università di Roma. Il sensore è in grado di rilevare automaticamente quando l'utente è alla guida di un'auto, utilizzando interazioni implicite e analisi contestuale dei dati forniti dalla radio Bluetooth degli smartphone, sfruttando le API di Android per la gestione delle connessioni Bluetooth. Il sistema calcola una "confidenza" relativa alla guida, che contribuisce al riconoscimento dello stato dell'utente tramite un algoritmo di media pesata tra vari sensori. I test effettuati hanno dimostrato l'affidabilità del sensore nello scenario reale, confermandone l'efficacia e l'integrazione con il sistema GeneroCity. Futuri sviluppi prevedono l'introduzione di modelli di machine learning per migliorare la determinazione dello stato dell'utente e l'implementazione di servizi esterni per il riconoscimento dei dispositivi Bluetooth.

\end{abstract}