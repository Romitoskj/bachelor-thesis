\begin{abstract}

La crescente urbanizzazione e il numero sempre più elevato di veicoli in circolazione rendono la gestione della mobilità urbana e dei parcheggi un tema centrale per le città moderne. GeneroCity è un'applicazione di smart parking sviluppata per Android e iOS dal Gamification Lab della Sapienza Università di Roma, il cui obiettivo è migliorare la gestione dei parcheggi urbani, riducendo il traffico causato dalla ricerca di un posto libero. L'app si basa su un sistema innovativo di interazioni implicite, progettato per rilevare automaticamente quando un utente lascia o cerca un parcheggio, minimizzando l'uso attivo dello smartphone e promuovendo la sicurezza alla guida.

Questa relazione presenta il mio lavoro svolto durante il tirocinio, ovvero lo sviluppo e l'integrazione di un sensore Bluetooth all'interno dell'applicazione GeneroCity. Il sensore utilizza le funzionalità Bluetooth degli smartphone Android per determinare con alta probabilità se un utente è alla guida, attraverso la rilevazione e l'analisi delle connessioni con dispositivi Bluetooth vicini, come autoradio. Viene illustrata la progettazione del sistema sensoriale, l'implementazione in Java tramite l'uso delle API di Android e la gestione della comunicazione tra il sensore e il server per l'aggiornamento dello stato dell'utente. L'approccio modulare adottato garantisce l'indipendenza tra i diversi sensori, con l'obiettivo di migliorare l'affidabilità del sistema e consentire futuri sviluppi basati su algoritmi di apprendimento automatico. Infine, sono discusse le potenzialità di ulteriori miglioramenti per aumentare l'accuratezza e la scalabilità del sistema.

\end{abstract}