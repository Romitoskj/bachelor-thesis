\chapter{Integrazione del sensore bluetooth in Generocity} \label{chap:Bluetooth-sensor}
In questo capitolo verrà descritto come il è stato integrato il sensore Bluetooth nel sistema precedentemente descritto. Esso utilizzerà alcune delle funzionalità e dettagli implementativi illustrati nel capitolo \ref{chap:app-separata} a cui verranno aggiunti degli algoritmi per rilevare la connessione di automobili e per il calcolo della confidenza.

Implementazioni di quanto detto in precedenza su Generocity con qualche piccola modifica.

Specificare anche la struttura dei package

\section{L'aggiunta della classe BluetoothSensor ed il nuovo Controller}
Estende la classe Sensor e ascolta i cambiamenti del controller bluetooth

\section{Algoritmo per il rilevamento di una macchina}

\section{La confidenza calcolata dal sensore e i dati inviati al server}
\[connectionScore(connectionCount) = \frac{25}{10000} * (1.4^{connectionCount-1} - 1)\]

\begin{figure}[t]
    \centering
    \caption{Incremento della confidenza basato sul numero di connessioni}
    \label{fig:confidence-func}
\begin{tikzpicture}
\begin{axis}[
    axis lines = left,
    width = 13cm,
    xmax=15,
    xlabel=numero di connessioni,
    xtick={0,...,16},
    ymax=1,
    ylabel=confidenza
]
  \addplot[
        blue,
        thick,
        domain= 1:15,
        samples=100
    ] {0.75 + 0.0025 * (1.4^(x-1) -1)};
\end{axis}
\end{tikzpicture}
\end{figure}

\section{La presentazione dei dati}
\section{Testing}